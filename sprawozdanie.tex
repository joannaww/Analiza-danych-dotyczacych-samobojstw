\documentclass[12pt]{mwrep}
\usepackage[utf8]{inputenc}
\usepackage{polski}
\usepackage{lmodern}
\usepackage{wasysym}
\usepackage[hidelinks]{hyperref}
\usepackage{array}
\newcolumntype{P}[1]{>{\centering\arraybackslash}p{#1}}
\newcolumntype{M}[1]{>{\centering\arraybackslash}m{#1}}
\author{Joanna Matuszak 255762, Joanna Wojciechowicz 255747}
\title{Sprawozdanie 1}
\date{\today}
\begin{Schunk}
\begin{Sinput}
> pdf.options(encoding='CP1250')
> library(xtable)
\end{Sinput}
\end{Schunk}

\usepackage{Sweave}
\begin{document}
\input{sprawozdanie-concordance}
\maketitle
\tableofcontents
\chapter{Wstęp}

Analizowane dane dotyczą ilości samobójstw w zależności od roku, płci i wieku w wybranych państwach świata. Zbiór danych pochodzi ze strony \href{https://www.kaggle.com/}{Kaggle} \url{https://www.kaggle.com/szamil/who-suicide-statistics} i został zebrany przez Światową Organizację Zdrowia WHO. 

Celem analizy będzie odpowiedź na pytanie „Kto częściej popełnia samobójstwa – kobiety czy mężczyźni?". Przeprowadzimy analizę w zależności od kraju i grupy wiekowej. Na początku skupimy się na analizie całościowej - weźmiemy pod uwagę cały zbiór danych, a następnie na reprezentatywnej grupie państw Europy Zachodniej.

Aby to zrobić dokonamy wstępnej obróbki danych, w której przeanalizujemy braki danych oraz błędne wartości w zbiorze. Następnie, wykorzystując różnego rodzaju wykresy, takie jak wykres słupkowy oraz wykres liniowy, wyciągniemy wnioski, które pozwolą odpowiedzieć na postawione pytanie badawcze.


\chapter{Wstępna obróbka danych}

\begin{Schunk}
\begin{Sinput}
> library(dplyr)
> library(ggplot2)
> df <- read.csv("who_suicide_statistics.csv", stringsAsFactors = TRUE)
\end{Sinput}
\end{Schunk}

Zbiór danych, którego kilka początkowych wierszy zostało przedstawionych w Tabeli \ref{tab:tabelka1}, zawiera zarówno zmienne kategoryczne jak i numeryczne.

